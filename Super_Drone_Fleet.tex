\documentclass[a4paper,12pt]{article}
\usepackage[utf8]{inputenc}
\usepackage{fancyhdr}
\usepackage{graphicx}
\graphicspath{ {./report_images/} }

\pagestyle{fancy}


\lhead{Control \#1924744}
\rhead{Page \thepage}

\begin{document}

\title{Insert Title Here}
\author{Control \#1924744}
\date{28th January 2019}
\maketitle
\newpage

\pagenumbering{roman}

\section*{\hfil Summary\hfil}
\begin{center}
\textit{"All models are wrong but some are useful''} \\*
\end{center}

\newpage

\section*{\hfil HELP INC MEMO\hfil}
\newpage


\tableofcontents
\newpage
\pagenumbering{arabic}


\newpage


\section{Introduction}
\subsection{Background}
Puerto Rico is a small US territory situated on the 18th parallel.
\subsection{Problem Restatement}
Our problem is divided into two main objectives;
\begin{itemize}
\item[-]Delivering required medical packages to the associated medical centres.
\end{itemize}


\section{The Assumptions}
The following core assumptions were made before embarking on our first model. These were necessary to fully understand the strategy we would
need to develop to distribute medical supplies and survey roads:

\begin{itemize}
\item[-]
\end{itemize}

\newpage

\section{The Ideal Setup}

\subsection{P-CB (Package to Cargo Bay) Configuration}

In order to understand which drone was suitable to use in deliveries it was necessary to begin with the core fundamentals
of how a cargo bay would store medical packages. In order to visualize this a table was generated for the
different combinations of medical packages each cargo bay could carry.\\*
In the table below the different maximum loads each cargo bay can hold is shown.\\*
$m_x =$ medical package $x$


\begin{center}
\begin{tabular}{ |c|l| }
 \hline
 Cargo Bay 1 & ${(m_1),(m_2),(m_3)}$  \\
 Cargo Bay 2 & ${(m_1),(m_2,m_2,m_2),(m_3,m_3),(m_1,m_2)(m_2,m_3)}$  \\
 \hline
\end{tabular}
\end{center}

As seen in the table cargo bay 1 is limited to sending one medical package per drone whereas cargo bay 2 has much
more flexible types of combinations available.
We will next look at the daily needs of each medical center and the associated CB deliveries that would be required.

\subsection{CB combinations for medical centres}
\begin{center}
\begin{tabular}{ |c|c|c|c| }
 \hline
 Medical Centre & Daily Need & C1 & C2 \\
  1 & $m_1,m_3$ & $(m_1),(m_3)$ & $(m_1),(m_2,m_3),(m_3,m_3)$  \\
  2 & $m_1,m_1,m_3$ & $(m_1),(m_2)$ & $(m_1),(m_2,m_3),(m_3,m_3)$  \\
  3 & $m_1,m_2$ & $(m_1),(m_2)$ & $(m_1),(m_2,m_3),(m_3,m_3)$  \\
  4 & $m_1,m_1,m_2,m_3,m_3$ & $(m_1),(m_2)$ & $(m_1),(m_2,m_3),(m_3,m_3)$  \\
  5 & $m_1$ & $(m_1),(m_2)$ & $(m_1),(m_2,m_3),(m_3,m_3)$  \\
 \hline
\end{tabular}
\end{center}

\subsection{D-C (Drone to Container) Configuration}



\newpage

\section{The Models}


\newpage

\section{Comparison Between Models}

\newpage

\section{Sensitivity Analysis}

\newpage

\section{Conclusions}

\newpage

\section{Appendices}


\end{document}
